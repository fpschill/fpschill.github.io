%% start of file `template_en.tex'.
%% Copyright 2007 Xavier Danaux (xdanaux@gmail.com).
%
% This work may be distributed and/or modified under the
% conditions of the LaTeX Project Public License version 1.3c,
% available at http://www.latex-project.org/lppl/.


\documentclass[10pt,a4paper,sans]{moderncv}

% moderncv themes
%\moderncvtheme[blue]{casual}
\moderncvtheme[grey]{classic}                 % optional argument are 'blue' (default), 'orange', 'red', 'green', 'grey' and 'roman' (for roman fonts, instead of sans serif fonts)

% character encoding
\usepackage[utf8]{inputenc}                   % replace by the encoding you are using

% adjust the page margins
\usepackage[scale=0.875]{geometry}
\recomputelengths                             % required when changes are made to page layout lengths

% personal data
\firstname{Prof. Dr. Frank-Peter}
\familyname{Schilling}
%\title{Curriculum Vitae}               % optional, remove the line if not wanted
\address{ZHAW}{CH-8401 Winterthur}    % optional, remove the line if not wanted
\mobile{+41 58 934 69 55}                    % optional, remove the line if not wanted
%\phone{+41 22 76 79267}                      % optional, remove the line if not wanted
%\fax{+41 22 7668866}                          % optional, remove the line if not wanted
\email{scik@zhaw.ch}                      % optional, remove the line if not wanted
\social[linkedin]{frankpeterschilling}
\homepage{fpschill.github.io}
%\extrainfo{additional information (optional)} % optional, remove the line if not wanted
\photo[60pt]{scik.jpg}                         % '64pt' is the height the picture must be resized to and 'picture' is the name of the picture file; optional, remove the line if not wanted
%\quote{Some quote (optional)}                 % optional, remove the line if not wanted

%\nopagenumbers{}                             % uncomment to suppress automatic page numbering for CVs longer than one page

%----------------------------------------------------------------------------------
%            content
%----------------------------------------------------------------------------------
\begin{document}

\maketitle

%\textit{Senior lecturer and group leader; Intl. research programme manager with a background in AI, Deep Learning, and Particle Physics; Co-Discoverer of the Higgs particle @ CERN; h-index 150}

\section{Employment}

\cventry{2023-}{Deputy Director}{Centre for AI (CAI), ZHAW}{Winterthur (CH)}{}{}

\cventry{2022-}{Adjunct Professor of Data Science}{Victoria University of Wellington}{Wellington (NZ)}{}{}

\cventry{2022-}{Senior Lecturer \& Group Leader}{ZHAW}{Winterthur (CH)}{}
{Head, Intelligent Vision Systems (IVS) group, Centre for AI (CAI); Computer Vision and MLOps
%	; Teaching (BSc/MSc); Head, continuing education at CAI; Coordinator ZHAW-UZH PhD Programme in Data Science
}

\cventry{2019-2022}{Senior Researcher}{ZHAW} {Winterthur (CH)}{}
{Centre for AI (CAI, 2021-); Inst. for Applied Information Technology (InIT, 2019-2021); AI and Deep Learning 
	%Teaching (BSc/MSc); project management; organization of conferences (ANNPR 2020, ISSDS 2021) and seminars (Datalab seminar); ZHAW digital advisory role (DIZH fellowships)
}

\cventry{2018-2019}{Guest Scientist}{Zurich University of Applied Sciences ZHAW} {Winterthur (CH)}{} 
{Institute for Applied Information Technology (InIT); AI/ML, Deep Learning, Computer Vision}

%
\cventry{2016-2018}{CEO}{fp solutions}{Bern and Winterthur (CH)}{}{}
%Software engineering}
%

\cventry{2007-2015}{Visiting Reseacher}{CERN}{Geneva (CH)}{}
{Permanently delegated from KIT to conduct research at CMS experiment at the LHC}
%
\cventry{2007-2014}{Senior Research Scientist}{Karlsruhe Institute of Technology KIT}{Karlsruhe (DE)}{}
{Research at the CMS experiment at the Large Hadron Collider LHC%; led research and software projects (planning and strategy, organization and follow-up, quality control / assurance, documentation, reporting); author and reviewer of scientific publications and technical reports; software development and cloud computing; big data and machine learning in the context of Petabyte data; member of various international scientific bodies; organization of intl. conferences and workshops
}
%
\cventry{2004-2006}{Research Fellow}{CERN}{Geneva (CH)}{}
{ Preparatory work for the LHC and the CMS experiment
	%; project management; software development; development of calibration algorithms and infrastructure; author of technical reports
}
%
\cventry{2001-2004}{Postdoctoral Fellow}{DESY}{Hamburg (DE)}{}
%{FH1 group, group leader: Dr. Eckhard Elsen}             
{Research and data analysis at the H1 experiment at the HERA particle accelerator
	%; project management; operations and data taking; software development; statistical data analysis; scientific publishing
}


\section{Education \& Training}
%

\cventry{2024}{IEEE CertifAIEd Authorized Lead Assessor}{IEEE SA}{}{}{}

\cventry{2023-2024}{Certificate of Advanced Studies (CAS), University Didactics}{PH Zurich}{}{}{}

%\cventry{2014-2018}{Various continuing education courses}{edX, Coursera}{}{}{Data science, machine learning, software development, cloud computing, international relations}

\cventry{2015}{IPMA Certification in Project Management}{VZPM}{}{}{}

\cventry{1998-2001}{PhD in Physics (Dr. rer. nat.)}{University of Heidelberg (DE)}{}{}{}
%{Thesis: {\em Diffractive Jet production in Deep-Inelastic ep Collisions at HERA} }
%	\newline Advisor: Prof. Dr. Franz Eisele}  % arguments 3 to 6 are optional
%
\cventry{1992-1998}{Diploma in Physics (MSc equiv.)}{University of Heidelberg (DE)}{}{}{}
%{Thesis: {\em Diffractive Dijet Production at HERA} }
%	\newline Advisor: Prof. Dr. Franz Eisele}  % arguments 3 to 6 are optional

\section{Awards \& Publications}%, Presentations}

\cventry{2013}{EPS HEP Prize}{as member of CMS collaboration at CERN, for the discovery of the Higgs boson}{}{}{}

%\cvline{2012}{Invited review  "Top quark physics at the LHC: the first two years", \href{https://www.worldscientific.com/doi/abs/10.1142/S0217751X12300165}{Int. J. Mod. Phys. A27 (2012) 1230016} }

\cventry{1998-}{More than 450 scientific publications, h-index of 160}{}{}{}
{For details, see \href{https://fpschill.github.io/publications/}{https://fpschill.github.io/publications/}}

%\cventry{1998-}{More than 50 keynote, conference and seminar presentations}{}{}{}
%{For full list, see \href{https://fpschill.github.io/talks/}{https://fpschill.github.io/talks/}}


\section{Research Grants}% and Scholarships}
%

\cvline{2022}{"certAInty", Innosuisse with CertX AG, 600 KCHF }

\cvline{2022}{"OSR4H", with Roche Diagnostics, 30 KCHF }

\cvline{2021}{"AC3T", Innosuisse with Varian Medical Systems, 785 KCHF}

\cvline{2020}{"ANNPR 2020", ZHAW Digital Futures Fund, 10 KCHF}


\cvline{2007-2014}{Contributor to several research grants for MEUR research group funding with the German BMBF}

%\cvline{2004}{EU Marie-Curie Intra-European Fellowship with Univ. Birmingham (UK), declined}
%
%\cventry{2001}{Bjorn H. Wiik Scholarship}{39th Intl. School of Subnuclear Physics, Erice (Italy)}{}{}{}
%
%\cvline{1998-2001}{PhD scholarship (Graduiertenkolleg), German Research Society (DFG) }

%\section{Management and Leadership}
%
\section{Project Management}

\cventry{2019-}{Project manager}{Applied Deep Learning}{ZHAW CAI}{}
{Project management for various 3rd party funded research projects in the domain of AI, Deep Learning, Computer Vision and MLOps; Financial planning and reporting; IP contracts}

\cventry{2010-2011}{Programme coordinator "Top Quark Physics"}{CMS experiment}{CERN}{}
{Led 1 out of 7 physics working groups (100+ members) of the collaboration; strategic planning, scheduling and execution of analysis of first LHC data; assessment and quality control; orchestrating of publications; reporting to management}

\cventry{2006-2007}{Project leader "Silicon tracker alignment"}{CMS experiment}{CERN}{}{}
%{Led a team of physicists and engineers; successful calibration of a large particle detector (16000 Si sensors with several 100 million read-out channels); development of innovative algorithms; preparation of software and computing infrastructure}

\cventry{2003-2004}{Trigger coordinator}{H1 experiment}{DESY}{}{}
%{Oversaw successful 24/7 operation of a central component of the particle detector as well as continuous optimization}

\cventry{2001-2003}{Physics group coordinator}{H1 experiment}{DESY}{}{}
%{Led team of scientists from different universities and institutes which analysed data of H1 experiment (matrix organisation); planning of resources; organisation of meetings; scientific evaluation and quality control; author and reviewer of scientific publications and reports; reporting to management}

%
\section{Research Infrastructure}

\cventry{2023-}{Expert group member}{Data Ethics, Big Data \& AI, data innovation alliance
}{}{}{}

\cventry{2022-}{Coordinator}{PhD Programme in Data Science between ZHAW and University of Zurich}{}{}{}

\cventry{2022-}{Head}{Continuing Education, ZHAW Centre for AI}{}{}{}

%\cventry{2022-}{Head of Studies}{Certificate of Advanced Studies (CAS) Machine Intelligence, ZHAW}{}{}{}

\cventry{2019-} {Main organizer}{CAI colloquium and Datalab seminar}{ZHAW}{}{Regular seminars on data science topics with internal and external speakers}

\cventry{2019-} {Advisory board member}{Breakout group DIZH fellowships}{ZHAW digital}{}{}
%{Major contributions to shaping of the DIZH fellowship programme}

\cventry{2011-2014}{Co-founder}{WG on Top Quark Physics at LHC}{LHC Physics Center at CERN (LPCC)}{}
{Transnational group of experts discussing scientific topics at the forefront of research in top quark physics}

\cventry{2008-2014}{Coordinator}{Top Quark Physics WG}{Helmholtz Alliance {\em Physics at the Terascale}}{}
{The “Terascale” Alliance bundles German activities in the field of high-energy collider physics. 
	%It is a network comprising all relevant German research institutes - 18 universities, two Helmholtz Centres and one Max Planck Institute
}

\cventry{2008-2012}{Advisory board member}{DESY scientific board (DESY WA)}{DESY}{}
{DESY WA is a committee which advises the DESY directorate in matters of research policy
	%, cooperation, strategic issuses as well as selection of senior research staff
}

%
\subsection{Workshop and Conference Organization}
\cvline{2021}{ISSDS 2021, Intl. Symposium on the Science of Data Science, ZHAW/Online (co-organizer)}
\cvline{2020}{ANNPR 2020, 9th IAPR Workshop on Artificial NN's in Pattern Recognition, ZHAW/Online (chair)}
\cvline{2011, 2012}{Helmholtz Alliance workshops on top quark physics, Wuppertal and Berlin (co-organizer)}
\cvline{2011}{TOP 2011, 4th Intl. workshop on top quark physics, Sant Feliu de Guixols (IAC member)}
%\cvline{2007}{CMS tracker alignment workshop, Hamburg (co-organizer)}
%\cvline{2004}{DIS 2004, 12th Intl. workshop on Deep-Inelastic Scattering, Slovakia (session organizer)}
%

% % % % % % % %

%\section{Research Experience}
%
%\subsection{AI and Deep Learning}

%\cvitem{2022-}{Project certAInty (CertX AG): Development of a certification scheme for safe and trustworthy AI}

%\cvitem{2022-}{Project SKACH (SKA Observatory): Deputy PM; Development of deep-learning based generative models for Radioastronomy}

%\cvitem{2022-}{Project OSR4H (Roche Diagnostics): PM; Development of deep-learning based open-set recognition methods for hematology}

%\cvitem{2022-}{Innosuisse Switzerland-South Korea project AC3T (Varian Medical Systems): Lead investigator and PM @ CAI; development of deep learning based solution for low dose (sparse) 3D and 4D CBCT reconstruction}

%\cvitem{2020-}{Innosuisse project DIR3CT (Varian Medical Systems): Lead investigator and PM @ CAI; development of deep learning based solution for mitigation of motion artefacts in reconstructed cone-beam computed tomography (CBCT) images}

%\cvitem{2018-2020}{Innosuisse project QualitAI: Development of a deep learning based computer vision solution for industrial quality control of medical devices (balloon catheters)%; deep convolutional neural networks; anomaly detection with Autoencoders/GANs}

%\subsection{Particle Physics with the CMS Experiment}

%\cvitem{2012-2014}{Higgs physics: Search for the Higgs in "boosted" $WH \rightarrow l\nu b\bar{b}$ mode using ML methods; Involved in experimental discovery of the Higgs particle \textbf{($>$19000 citations)}}

%\cvitem{2007-2011}{Top quark physics: major role in re-discovery of top quark at LHC; leading initial top quark  measurements in lepton+jets topologies; first journal publication of CMS using this channel}

%\cvitem{2004-2009}{Preparations for data taking: Major contrbutions to identification of b-jets, track reconstruction, tracker alignment (software development, calibrations)}


%\subsection{Particle Physics with the H1 Experiment}

%\cvitem{2001-2004}{Main analyzer, measurement of the inclusive, triple differential cross section of diffractive DIS processes with unprecedented precision; determination of NLO PDFs}

%\cvitem{1998-2001}{Primary analyzer, measurement of differential cross sections of diffractive dijet production; detailed comparisons with QCD models}


\section{Recent Teaching \&  Student supervision}
%

\subsection{Head of studies}


\cvline{2023-}{Creation of new modules: MLOps, Computer Vision  with Deep Learning, Generative AI in Teaching, CAS Advanced ML and MLOps}
\cvline{2022-}{Head of studies: MLOps, Computer Vision with Deep Learning, Generative AI in Teaching, CAS Advanced ML and MLOps, CAS Machine Intelligence}

\subsection{Teaching}

\cvline{2024-}{"MLOps" (BSc and Cont. Education), ZHAW, Spring 2024, Fall 2024}
\cvline{2024-}{"Computer Vision with Deep Learning" (BSc), ZHAW, Spring 2024}
\cvline{2022-}{"Deep Learning" (CAS Machine Intelligence), ZHAW, Fall 2022, 2023, 2024}
\cvline{2019-}{"Artificial Intelligence 1" (BSc), ZHAW, Fall 2019, 2020, 2021, 2022}
%\cvline{2021}{"Machine Learning", STFW Winterthur, Spring 2021}
\cvline{2019}{"Machine Intelligence Lab" and "AI Seminar"  (MSc), ZHAW, Fall 2019}

%\cvline{2019}{"Artificial Intelligence Seminar" (MSc), ZHAW, Spring 2019}

%\cvline{2008}{Computer exercises for lecture on advanced particle physics, KIT}
%\cvline{2007}{Seminar on LHC physics and key experiments of particle physics, KIT}
%\cvline{2003}{Lecturer, CTEQ summer school on QCD analysis and phenomenology, Spain}
%\cvline{1997-2000}{Tutor \& instructor, physics lectures \& labs (Univ. Heidelberg)}

\subsection{Student supervision}
%
%\cvline{2022-}{Supervision of a PhD student (ZHAW)}
\cvline{2001-}{Supervision of 10 PhD students (AI/ML, computer vision; QCD, Top quark and Higgs physics)}
\cvline{2021-}{Supervision of several MSc/MAS students (AI/ML, computer vision)}
%\cvline{2005}{Supervision of a CERN summer student (CERN; MS commissioning)}
%\cvline{2000}{Supervision of a MSc student (DESY; QCD physics)}
%

% % % % % % % %5

%\section{Outreach}
%
%\cvline{2007-2014}{Guided tours of CERN and CMS, including 2008 CERN open days}
%\cvline{2008}{Guide at the LHC exhibition {\em Weltmaschine}, Berlin (Germany)}
%\cvline{2008}{CMS experiment guide during CERN open day}
%\cvline{2005}{Translation of public CERN web pages into German language}


%\section{Schools and Training}
%
%\cvline{2001}{39th Intl. School of Subnuclear Physics, Erice (Italy)}
%
%\cvline{2000}{CTEQ school on QCD analysis and phenomenology, Lake Geneva (USA)}
%
%\cvline{1998}{Management consultancy courses on project management, presentation and moderation}
%
%\cvline{1998}{Autumn school for High Energy Physics, Maria Laach (Germany)}
%
%\cvline{1995}{Summer Student at DESY: Top quark   production at a  linear collider; Adivsor: Dr. J.~Meyer}

%\section{Networks and Reviewing}
\section{Networks}
%\cvline{}{ZHAW Datalab, ZHAW Digital Futures Lab, ZHAW Digital Health Lab}
%\cvline{}{CLAIRE, ELLIS (supporter)}
\cvline{}{data innovation alliance, European Physical Society (EPS), German Physical Society (DPG), CLAIRE, ELLIS (supporter), ZHAW Datalab, ZHAW Digital Futures Lab, ZHAW Digital Health Lab}
	%, ZHAW Datalab, ZHAW Digital Futures Lab, ZHAW Digital Health Lab}
%\cvline{1997-present}{German Physical Society (DPG)}
%
\section{Journals}
\cvline{}{Reviewer/PC member for EPIA 2022, CVPR 2022, AutoML 2022, ANNPR 2022, ISSDS 2021, ANNPR 2020}
	%, IJCAI 2019}
%\cvline{}{Guest editor for MDPI Computers, MDPI Imaging}
\cvline{}{Reviewer/Guest Editor for MDPI (several), Phys.Lett.B, Eur.Phys.J. C, JHEP, Int.J.Mod.Phys. A}
%\cvline{}{Internal reviewer within CMS, H1 collaborations}

%\section{IT Competences}
%\cvline{Languages}{PYTHON, C++, C\#, FORTRAN}
%\cvline{Software}{Pytorch, Tensorflow, Keras, ROOT, Docker, Oracle, SQL, Openstack, Git}
%%\cvline{OS/Office}{Linux (admin level), Windows, MS Office, HTML/CSS, Wikis, CMS}

\section{Languages}
%\cvline{}{Experienced in software development in C++, PYTHON, PERL, FORTRAN using SVN/CVS}
%\cvline{}{Experienced with GRID computing, client/server environments and DB applications}
%\cvline{}{Proficient in web design in HTML/CSS and using content management systems}
%
%\section{Languages}
\cvline{}{German (native); English (business fluent); French (very good)}
%\cvline{German}{native}
%\cvline{English}{business fluent, daily use in the professional environment since >20 years}
%\cvline{French}{very good, lived for 11 years in \textit{Suisse Romande}}

\end{document}
